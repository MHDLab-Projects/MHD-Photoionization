\section{Supplemental Results}

\subsection{Experiment Overviews}

Below are overviews of the experiments performed. The colored regions are test case sequences. Times are PST. 

\begin{figure}[h]
    \includegraphics[width=0.8\linewidth]{\repodir/experiment/analysis/auto/output/figures/tc_plot_2023-04-07.png} 
    \caption{Experiment Overview for 2023-04-07 }
    \label{fig:expt_overview_2023-04-07}
\end{figure}


\begin{figure}[h]
    \includegraphics[width=0.8\linewidth]{\repodir/experiment/analysis/auto/output/figures/tc_plot_2023-05-12.png} 
    \caption{Experiment Overview for 2023-05-12 }
    \label{fig:expt_overview_2023-05-12}
\end{figure}


\begin{figure}[h]
    \includegraphics[width=0.8\linewidth]{\repodir/experiment/analysis/auto/output/figures/tc_plot_2023-05-18.png} 
    \caption{Experiment Overview for 2023-05-18}
    \label{fig:expt_overview_2023-05-18}
\end{figure}

\begin{figure}[h]
    \includegraphics[width=0.8\linewidth]{\repodir/experiment/analysis/auto/output/figures/tc_plot_2023-05-24.png} 
    \caption{Experiment Overview for 2023-05-24}
    \label{fig:expt_overview_2023-05-24}
\end{figure}

\clearpage

\subsection{Equivalence Ratio Dependence}

On 2023-05-24, we performed measurements with equivalence ratio varying from the standard value of 0.8. 

\begin{itemize}
    \item 516 - a position sweep at fixed equivalence ratio 0.6
    \item 5x3\_pos - a position and equivalence ratio sweep at 5 equivalence ratios and 0.1\% potassium mass fraction. The lower potassium mass fraction allowed for longer times on each test case and therefore AES nK measurements. 
    \item 5x6\_pos - a position and equivalence ratio sweep at 5 equivalence ratios and 1\% potassium mass fraction. The high potassium mass fraction meant short times on each test case with only enough time for MWS measurements. 
\end{itemize}

Figure \ref{fig:5x3_pos_nK_m3} shows the AES measurements for the 5x3\_pos test case. Note the calibration of the AES signal for this test case exhibited significant calibration drift (see \ref{fig:5x3_pos_alpha}), and systematic error may be present in the nK values. 

\begin{figure}[]
\centering
\includegraphics[width=0.8\textwidth]{\repodir/experiment/analysis/tcs/output/figures/5x3_pos_nK_m3.png}
\caption{AES measurements 5x3\_pos}
\label{fig:5x3_pos_nK_m3}
\end{figure}

\begin{figure}[]
\centering
\includegraphics[width=0.8\textwidth]{\repodir/experiment/analysis/tcs/output/figures/5x3_pos_alpha.png}
\caption{absorption data for AES measurements 5x3\_pos. The absorption offset at high wavelength is due to drift in calibration.}
\label{fig:5x3_pos_alpha}
\end{figure}


\begin{figure}[]
\centering
\includegraphics[width=0.8\textwidth]{\repodir/experiment/analysis/tcs/output/figures/5x3_pos_mws_KOH.png}
\caption{MWS measurements 5x3\_pos. CFD centerline values for KOH are also shown.}
\label{fig:5x3_pos_mws_KOH}
\end{figure}


\begin{figure}[]
\centering
\includegraphics[width=0.8\textwidth]{\repodir/experiment/analysis/tcs/output/figures/5x6_pos_mws_KOH.png}
\caption{MWS measurements 5x6\_pos. CFD centerline values for KOH are also shown.}
\label{fig:5x6_pos_mws_KOH}
\end{figure}

