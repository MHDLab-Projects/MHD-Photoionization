
% Figure \ref{fig:diagnostic_processing}B shows the time profile of the AS signal at the main condition (536 goldi) on a logarithmic scale. In general, we observe a region of changing slope within the first few $\mu s$ after the laser pulse, followed by an exponential decay. After approximately 20-30 $\mu s$, the AS signal becomes dominated by background fluctuations (discussed in results section), therefore data is only fitted before this time (time is variable based on condition). 

We fit the $AS (t)$ profile with a kinetic model of the electron recombination process. We first normalize $AS (t)$ by its maximum value at the laser pulse, defining $AS_{norm} = AS(t)/max(AS(t))$. We assume that $AS_{norm}(t) \propto \Delta n_e (t)$, where $\Delta n_e (t)$ is the time dependent change in the free electron density from the steady state concentration. 

We fit $AS_{norm}$ by solving the following differential equation: 

\begin{equation}
    \label{eq:fit_eq}
    \frac{d\Delta n_e (t)}{dt} = - k_{r, m, eff} \Delta n_e (t) - k_{r, bm, K+}\Delta{n_e (t)}^2
\end{equation}

where $k_{r, m, eff} [1/s]$ is an effective monomolecular recombination rate coefficient, and $k_{r, bm, K+} [cm^3/particle/s]$ is the bimolecular recombination rate coefficient for capture by K+. $k_{r, m, eff}$ arises from electron capture from a variety of species. The derivation of this equation is shown in the SI.

In the fitting process, the differential equation is numerically solved and the solution is normalized to have the have the same form as $AS_{norm}$. This solution is performed iteratively in a least squares optimization with the free fit parameters being $k_{r,m,X}$ and $\Delta{n_e,0}$, where $\Delta{n_e,0}$ is the initial excited electron density. for $k_{r, bm, K+}$ we use the value obtained by Kelley and Padley 1973 (TODO).

% Figure \ref{fig:diagnostic_processing}B shows a fit of the AS signal at the main condition (536 goldi). In this plot, $AS_{norm}$ and fit have been converted to $\Delta n_e (t)$ by multiplying by $\Delta{n_e,0}$. 

