\section{Discussion}

We now discuss the model for the photoionization and recombination process. We tentatively propose a model:

1. KOH + $h\nu$ -> K+ + OH + e-

2. e- + O2 -> O2-

The details of step 1 are uncertain. The important point is that light absorption is through KOH, and not K. It is possible that other species contribute to the electron production through similar chemionization processes. However, KOH absorption is known to produce excited K atoms. [] That work showed that those excited K atoms decay through rapid (10ns) photoluminescence, but it is possible that chemiionization process occurs in parallel to the photoluminescence. Possibly, there is a collision with the excited K atoms that produces the e-. From Alkemade et al. we know that the K +M -> K+ + e- +M depends on a collision with a third body, and appears to be dependent on the potassium atom being in an excited state (known as the 'ladder model'). []



\begin{outline}
\1 Evidence of KOH -> e-, instead of K -> e-:
    \2 The photoluminescence lifetime, measured by the ICCD Camera (see Figure S\ref*{fig:SI_536_iccd}) consistent with UV induced chemiluminescence of KOH that has been observed in the literature. The photoluminescence is measured far from the center of the free jet, where the potassium density is low.
    \2 The UV absorption cross section of KOH is much larger than other major species present (see SI).
    \2 The AS signal linearly increases with the nominal potassium concentration, even though the potassium density measurements at the goldilocks region decreases. This   is dependent on the potassium concentration in the barrel, but not the goldilocks region. The measured nK However, CFD prediction of KOH concentrations scales with the AS signal. The photodiode measurement also scaled with the AS signal. We assume the photodiode signal is proportional to the chemiluminescence of KOH. We cannot measure the time decay of the chemiluminescence with the photodiode due to the bandwidth of the photodiode.
    \2 The MWS and photodiode signals peak downstream of the exit. The CFD data predict a peak in KOH in a similar location (SI). However, the difference between the barrel exit and this peak location is not as large as the experimental data. TODO: why is barrel KOH high. 

\1 Evidence for e- + O2 -> O2-
    \2 the decay time predicted for oxygen capture is the closest to the measured decay time. There is a significant uncertainty in the recombination coefficient for this process and our values fall within the range presented in the literature. 
    \2 The decay time appears to be independent of the position along the free jet. The decay time is difficult to quantify outside of the goldilocks region due to the AS fluctuations, but the AS magnitude is still appreciable after a few microseconds regardless of position. a similar order of magnitude to the exponential fit time constants obtained in the goldilocks region. CFD simulations (SI) show that the oxygen concentration is relatively constant along the free jet. This constant oxygen concentration of ~20\% occurs from running fuel lean, and air entrainment. 
    \2 Related, we have observed dependence of decay time on equivalence ratio (SI). However, similar to the position dependence, quantification of the decay time is difficult due to the AS fluctuations. Additionally, the profile of the torch changes with equivalence ratio (extends for high equivalence ratio), so it is difficult to deconvolve the effects of the profile change from the effects of the equivalence ratio.

\end{outline}



\subsection{Viability}

with this model, we examine the viability of photoionization for MHD power enhancement. 


\begin{figure}[h]
    \centering
    \includegraphics[width=\linewidth]{\repodir/final/figures/output/viability.png} 
    \caption{Figure of merit for photoionization for various temperatures and pressures. The line}   
    \label{fig:viability_alpha}
\end{figure}

Figure \ref{fig:viability_alpha} shows $\alpha_{PI}$ for various temperatures and pressures.  For $\alpha_{PI} > 1$ at low temperatures and pressures (blue region), the photoionization process exhibits a net energy return and is viable. For $\alpha_{PI} < 1$, at high temperatures and pressures (red region) the photoionization process is not viable. The line $\alpha_{PI} = 1$ is the boundary between these two regions. To determine the boundary line (green dashed  line), we interpolate $\alpha_{PI}$ each temperature to determine the pressure at which $\alpha_{PI} = 1$. These boundary lines are then shown for various conditions in the results section.

